\documentclass{article}
\usepackage[utf8]{inputenc}
\usepackage{subcaption}
\usepackage{amsmath}
\usepackage{amssymb}
\usepackage{hyperref}
\usepackage{titlesec}
\usepackage{xcolor}
\usepackage{fancyhdr}
\usepackage{graphicx}
\usepackage{multirow}
\usepackage[rightcaption]{sidecap}
\usepackage{verbatim}
\usepackage[backend=bibtex]{biblatex}
\addbibresource{report.bib}

\usepackage [ a4paper , hmargin =1.2 in , bottom =1.5 in ] { geometry }
\hypersetup{
    colorlinks=true,
    linkcolor=blue,
    filecolor=magenta,      
    urlcolor=cyan,
}


% Add header and footer code here
\pagestyle{fancy}
\lhead{CS104 Project on Spreadsheet Programming}
\rhead{Arnav Garg}
\cfoot{Page \thepage}
\title{CS104 Project \\ Spreadsheet Programming}
\author{Arnav Garg\\22B1021}
\date{}

\begin{document}
\maketitle
\tableofcontents
\clearpage

%code of section 1, with lists
\section{Introduction}
This report aims to provide a comprehensive description of the CS104 project on Spreadsheet Programming.\\
Here we implement a software to read and edit Google Sheets API to create a static system to deal with responses received through Google Forms.
%code of section 2, make appr
\section{Project Overview}
%para
We here implement a system for booking appointments at a multi locational health centre named Hope clinic. \\Form users have the option to fill in their time preference and their preferred location and they will be assigned a time slot based on their sex and chosen location. \\It also implements an algorithm to give a preference of slots to the users who fill the form first.\footnote{I also tried and played around with optimization of the assignment algorithm but was unable to implement it efficiently within constraints.}
\section{Basic Structure}
There are 10 queues in total- 5x2 for location and sex. I have implemented a system to prevent duplicacy by assigning a token number of 0 which is indicative to my code to send a mail to the form filler informing them of the redundancy.
The Google Sheets are read through a service account using the Google Sheets API\autocite{1} implementing the \texttt{gspread}\autocite{4} library in python.
I then account for 7 slots-
\begin{enumerate}
    \item 10AM-10:30AM
    \item 10:30AM-11AM
    \item 11AM-11:30AM
    \item 11:30AM-12PM
    \item 12PM-12:30PM
    \item 12:30PM-1PM
    \item 1PM-1:30PM
\end{enumerate}
with each slot corresponding to one token number.\\
Now I establish a SMTP connection to the GMail host \autocite{2} and send out auto generated emails using inbuilt libraries which specify the details of the appointment assigned to the user with \texttt{html} formatting.
\section{Customizations}
\subsection{Location specificity}
My thought process for the project went as follows- when a person books an appointment at a chain of clinics widely distributed across the country they can choose a specific location for which they are available to attend as locations are pretty spread out. Thus we can safely maintain location specific queues with a person able to book an appointment for only one possible location.
\subsection{Gender queues}
Implementing a sex based independent queue for both males and females to streamline the process of obtaining treatment is a natural consequence for distributed clinics addressing problems specifics to both the sexes. I have 2 queues at each of my 5 locations - making a total of 10 queues with each of them having a unique token number system starting from 1.\\
A token number of 0 indicates duplicacy in token filling.
\subsection{Optimization algorithm and time preference}
I have added an option for the form filler to choose multiple time slots out of the 7 given to indicate their availability and assist us in assigning a time slot favourable to them. However this poses a problem, while we can trivially assign a method for assignment, by just assigning the first or last selected slot, it raises problems of optimization. This can be shown by taking examples for as less as 3 users and 4 slots. This dips into the realm of \textbf{matching theory of bipartite graphs}\autocite{3} which turned out to be tough to implement. However an efficient algorithm was developed but not implemented to keep time and space complexities feasible for very large values.\\\\
\textbf{Claim:} An example of an efficient algorithm is assignment in \textbf{order for which slot has the least takers}, then \textbf{assignment by ascending order of token number} and then finally in a trivial order. 

\subsection{Email modification}
I have modified the bodies through the \texttt{MIMEText} function of my email to parse \texttt{html}\autocite{6}, so that my emails can display formatted text. I also added a check for age which prompts minors to be accompanied by a guardian and senior citizens to avail a special discount.
\section{Important Links}
\begin{itemize}
    \item \href{https://forms.gle/KnfyxSyh6hvjdoBu9}{Google Form}
    \item \href{https://docs.google.com/spreadsheets/d/16OCCYVb8J6uGF6t1GXo0IHV1P57_r_8tpJccRHVZVUA/edit?usp=sharing}{Google Sheets}
    \item \href{https://github.com/Hal-lucin-ogen8/SpreadsheetToken}{GitHub Repo}
\end{itemize}

\nocite{5}
\nocite{7}
\printbibliography
\end{document}